\documentclass[man]{apa6}

\usepackage{amssymb,amsmath}
\usepackage{ifxetex,ifluatex}
\usepackage{fixltx2e} % provides \textsubscript
\ifnum 0\ifxetex 1\fi\ifluatex 1\fi=0 % if pdftex
  \usepackage[T1]{fontenc}
  \usepackage[utf8]{inputenc}
\else % if luatex or xelatex
  \ifxetex
    \usepackage{mathspec}
    \usepackage{xltxtra,xunicode}
  \else
    \usepackage{fontspec}
  \fi
  \defaultfontfeatures{Mapping=tex-text,Scale=MatchLowercase}
  \newcommand{\euro}{€}
\fi
% use upquote if available, for straight quotes in verbatim environments
\IfFileExists{upquote.sty}{\usepackage{upquote}}{}
% use microtype if available
\IfFileExists{microtype.sty}{\usepackage{microtype}}{}

% Table formatting
\usepackage{longtable, booktabs}
\usepackage{lscape}
% \usepackage[counterclockwise]{rotating}   % Landscape page setup for large tables
\usepackage{multirow}		% Table styling
\usepackage{tabularx}		% Control Column width
\usepackage[flushleft]{threeparttable}	% Allows for three part tables with a specified notes section
\usepackage{threeparttablex}            % Lets threeparttable work with longtable

% Create new environments so endfloat can handle them
% \newenvironment{ltable}
%   {\begin{landscape}\begin{center}\begin{threeparttable}}
%   {\end{threeparttable}\end{center}\end{landscape}}

\newenvironment{lltable}
  {\begin{landscape}\begin{center}\begin{ThreePartTable}}
  {\end{ThreePartTable}\end{center}\end{landscape}}

  \usepackage{ifthen} % Only add declarations when endfloat package is loaded
  \ifthenelse{\equal{\string man}{\string man}}{%
   \DeclareDelayedFloatFlavor{ThreePartTable}{table} % Make endfloat play with longtable
   % \DeclareDelayedFloatFlavor{ltable}{table} % Make endfloat play with lscape
   \DeclareDelayedFloatFlavor{lltable}{table} % Make endfloat play with lscape & longtable
  }{}%



% The following enables adjusting longtable caption width to table width
% Solution found at http://golatex.de/longtable-mit-caption-so-breit-wie-die-tabelle-t15767.html
\makeatletter
\newcommand\LastLTentrywidth{1em}
\newlength\longtablewidth
\setlength{\longtablewidth}{1in}
\newcommand\getlongtablewidth{%
 \begingroup
  \ifcsname LT@\roman{LT@tables}\endcsname
  \global\longtablewidth=0pt
  \renewcommand\LT@entry[2]{\global\advance\longtablewidth by ##2\relax\gdef\LastLTentrywidth{##2}}%
  \@nameuse{LT@\roman{LT@tables}}%
  \fi
\endgroup}


  \usepackage{graphicx}
  \makeatletter
  \def\maxwidth{\ifdim\Gin@nat@width>\linewidth\linewidth\else\Gin@nat@width\fi}
  \def\maxheight{\ifdim\Gin@nat@height>\textheight\textheight\else\Gin@nat@height\fi}
  \makeatother
  % Scale images if necessary, so that they will not overflow the page
  % margins by default, and it is still possible to overwrite the defaults
  % using explicit options in \includegraphics[width, height, ...]{}
  \setkeys{Gin}{width=\maxwidth,height=\maxheight,keepaspectratio}
\ifxetex
  \usepackage[setpagesize=false, % page size defined by xetex
              unicode=false, % unicode breaks when used with xetex
              xetex]{hyperref}
\else
  \usepackage[unicode=true]{hyperref}
\fi
\hypersetup{breaklinks=true,
            pdfauthor={},
            pdftitle={Day by Day, Hour by Hour: Naturalistic Language Input to Infants},
            colorlinks=true,
            citecolor=blue,
            urlcolor=blue,
            linkcolor=black,
            pdfborder={0 0 0}}
\urlstyle{same}  % don't use monospace font for urls

\setlength{\parindent}{0pt}
%\setlength{\parskip}{0pt plus 0pt minus 0pt}

\setlength{\emergencystretch}{3em}  % prevent overfull lines


% Manuscript styling
\captionsetup{font=singlespacing,justification=justified}
\usepackage{csquotes}
\usepackage{upgreek}



\usepackage{tikz} % Variable definition to generate author note

% fix for \tightlist problem in pandoc 1.14
\providecommand{\tightlist}{%
  \setlength{\itemsep}{0pt}\setlength{\parskip}{0pt}}

% Essential manuscript parts
  \title{Day by Day, Hour by Hour: Naturalistic Language Input to Infants}

  \shorttitle{Day by Day, Hour by Hour}


  \author{Elika Bergelson\textsuperscript{1,2}, Andrei Amatuni\textsuperscript{1,2}, Shannon Dailey\textsuperscript{1,2}, Sharath Koorathota\textsuperscript{2,3}, \& Shaelise Tor\textsuperscript{2,4}}

  % \def\affdep{{"", "", "", "", ""}}%
  % \def\affcity{{"", "", "", "", ""}}%

  \affiliation{
    \vspace{0.5cm}
          \textsuperscript{1} Duke University\\
          \textsuperscript{2} University of Rochester\\
          \textsuperscript{3} Columbia University Medical Center\\
          \textsuperscript{4} Syracuse University  }

  \authornote{
    Elika Bergelson, Psychology \& Neuroscience, Center for Cognitive
    Neuroscience, Duke University; Center for Developmental Science
    
    Andrei Amatuni, Psychology \& Neuroscience, Duke University
    
    Shannon Dailey, Psychology \& Neuroscience, Duke University
    
    Sharath Koorathota, Columbia University Medical Center
    
    Shaelise Tor, Marriage and Family Therapy, Syracuse University
    
    N.B.: all authors were in Brain \& Cogsci at U. Rochester during data
    collection and have no COI to declare
    
    Correspondence concerning this article should be addressed to Elika
    Bergelson, 417 Chapel Drive, Box 90086. E-mail:
    \href{mailto:elika.bergelson@duke.edu}{\nolinkurl{elika.bergelson@duke.edu}}
  }


  \abstract{Measurements of infants' quotidian experiences provide critical
information about early development. However, the role of sampling
methods in providing these measurements is rarely examined. Here we
directly compare language input from hour-long video-recordings and
daylong audio-recordings within the same group of 44 infants at 6 and 7
months. We compared 12 measures of language quantity and lexical
diversity, talker variability, utterance-type, and object presence,
finding moderate correlations across recording-types. However,
video-recordings generally featured far denser noun input across these
measures compared to the daylong audio-recordings, more akin to `peak'
audio hours (though not as high in talkers and word-types). Although
audio-recordings captured \textasciitilde{}10 times more awake-time than
videos, the noun input in them was only 2--4 times greater. Notably,
whether we compared videos to daylong audio-recordings or peak audio
times, videos featured relatively fewer declaratives and more questions;
furthermore, the most common video-recorded nouns were less consistent
across families than the top audio-recording nouns were. Thus, hour-long
videos and daylong audio-recordings revealed fairly divergent pictures
of the language infants hear and learn from in their daily lives. We
suggest short video-recordings provide a `dense and somewhat different'
sample of infants' language experiences, rather than a `typical' one,
and should be used cautiously for extrapolation about common words,
talkers, utterance-types, and contexts at larger timescales. If theories
of language development are to be held accountable to `facts on the
ground' from observational data, greater care is needed to unpack the
ramifications of sampling methods of early language input.}
  \keywords{language acquisition, naturalistic observational data, infants, early
home environment, language input, cognitive development \\

    \indent Word count: 3949
  }





\usepackage{amsthm}
\newtheorem{theorem}{Theorem}
\newtheorem{lemma}{Lemma}
\theoremstyle{definition}
\newtheorem{definition}{Definition}
\newtheorem{corollary}{Corollary}
\newtheorem{proposition}{Proposition}
\theoremstyle{definition}
\newtheorem{example}{Example}
\theoremstyle{definition}
\newtheorem{exercise}{Exercise}
\theoremstyle{remark}
\newtheorem*{remark}{Remark}
\newtheorem*{solution}{Solution}
\begin{document}

\maketitle

\setcounter{secnumdepth}{0}



\section{Highlights}\label{highlights}

\begin{itemize}
\tightlist
\item
  We measured 44 infants' early noun input during free-form interactions
  in hour-long videos and daylong audio-recordings; sampling approach
  shifted potential conclusions about home language environment.
\item
  Across quantity, utterance-type, object presence, and talker measures,
  nouns-per-minute were 2--4 times more frequent in video- than
  audio-recordings; videos were similar to \emph{peak} audio hours.
\item
  Nouns in videos occurred relatively more often in questions and less
  often in declaratives than they did in daylong or peak audio-recording
  hours
\item
  The most frequent nouns in daylong and peak audio-recording hours
  highly overlapped in identity and across families; this was less true
  for top video nouns.
\end{itemize}

Researchers have long studied development by observing infants in their
natural habitats (Taine, 1876; Williams, 1937). Over the past 20--30
years, written records have been increasingly supplemented with audio-
and video-recordings, depicting infants' linguistic, social, and
physical landscape. Such data --- often shared through repositories like
CHILDES and Databrary --- in turn provide a proxy for various
\enquote{input} measures in theories of social, motor, and particularly,
\emph{linguistic} development (MacWhinney, 2001).

Furthermore, recent technological advances have harnessed longer and
denser recordings to study infants' input and language skills (Bergelson
\& Aslin, 2017; Oller et al., 2010; B. C. Roy, Frank, DeCamp, Miller, \&
Roy, 2015; VanDam et al., 2016; Weisleder \& Fernald, 2013, \emph{inter
alia}). Such naturalistic data aim to reveal what infants learn from
while exploring their biological endowments and environmental resources.

However, wider-ranging technology creates more decision-points.
Researchers must decide on recording modalities (e.g.~audio, video),
where, whom, and how long to record, and whether to capture structured
or free-ranging interactions, with or without experimenters present. The
equivalence of these decisions is rarely tested. Problematically, this
leads to research with \emph{theoretical} conclusions built on
unmeasured \emph{methodological} assumptions.

Recently, Tamis-LeMonda, Kuchirko, Luo, Escobar, and Bornstein (2017)
directly compared sampling methods by analyzing mother-infant behavior
in 5-minute structured interactions and 45 minutes of free play. They
found that while language quantity across contexts correlated, infants
experienced more words per minute in structured interactions than in
free play. They conclude that sampling must match the research question,
cautioning that extrapolations from short samples merit extra care.

In contrast, work by Hart and Risley (1995) extrapolated extensively.
Based on 30 recorded hours per family (collected over 2.5 years), they
estimated that by age four, children receiving public assistance (n=6)
heard \textgreater{}30-million fewer words than professional-class
children (n=13). While their results merited and received follow-up
(e.g. Fernald, Marchman, \& Weisleder, 2013; Noble, Norman, \& Farah,
2005, \emph{inter alia}), they have also been criticized as extreme
over-extrapolation (Dudley-Marling \& Lucas, 2009; Michaels, 2013).

Still other research analyzes base rates of certain linguistic phenomena
through child corpora (Brent \& Siskind, 2001; Lidz, Waxman, \&
Freedman, 2003; Tomasello, 2000). Unfortunately, predetermining
\enquote{appropriate} sampling for such base rates is difficult. For
instance, practically any length of adult speech, will find function
words (e.g. \enquote{of}) at much higher rates than content words (e.g.
\enquote{fork}). For many questions, however, potential sampling bias is
unknown, leaving practical constraints to guide sampling parameters.

We explore sampling directly, comparing hour-long video-recordings and
daylong audio-recordings in a single sample of 44 infants, as part of a
larger study on early noun learning. We annotated concrete nouns said to
infants, focusing on nouns given their prevalence in the early
vocabulary (Dale \& Fenson, 1996). We further annotated three properties
previously linked with early language learning: utterance-type, which
provides syntactic/situational information (Brent \& Siskind, 2001;
DeBaryshe, 1993; Hoff \& Naigles, 2002), object presence
(i.e.~referential transparency), which tags whether spoken words'
referents are present and attended to (Bergelson \& Aslin, 2017;
Bergelson \& Swingley, 2013; Cartmill et al., 2013; Yurovsky, Smith, \&
Yu, 2013), and talker, which measures speaker quantity and prevalence
(Bergmann, Cristia, \& Dupoux, 2016; Rost \& McMurray, 2010).

This design sets up two overarching questions. First, does noun input in
one video-recorded hour predict noun input in an audio-recorded day?
Second, do input quantities differ once time is normalized? If the input
is equivalent and predictive across recording-types, then observational
data-collection approaches can vary with impunity. If not, understanding
methodological biases can help learning theories incorporate appropriate
bounds on data quantity and variability.

Thus, we examine home recordings across four key properties of language
input: word quantity, utterance-type, object presence, and talker. This
seemingly methodological investigation has deep implications for
developmental theory: we examine how sampling may alter conclusions
about the linguistic input driving early development.

\section{Methods}\label{methods}

\subsection{Participants}\label{participants}

Infants were recruited from a database of local families. Forty-six
participants enrolled; two dropped out leaving 44 in the final sample.
All were full-term (40±3 weeks), had normal vision and hearing, and
heard \(\geq 75\%\) spoken English. Participants were 95\% white; 75\%
of mothers had \(\geq\)B.A. Families were enrolled in a yearlong study
that included monthly audio- and video-recordings, as well as in-lab
visits every other month. See Table \ref{tab:recording-ages-table} for
age details. Here we report on the home recording data from the first
two timepoints (6 and 7 months) of this study, for which participants
received
\$10.\footnote{We used these timepoints because infants had not yet begun producing words themselves (which changes the input). Given the broader project aims, these timepoints alone had the entire daylong audio-recording annotated.}

\subsection{Procedures}\label{procedures}

Participants gave consent at an initial lab visit for the larger study
through a University IRB approved process. Questionnaires concerning
participant background, not germane here, are reported elsewhere
(Bergelson \& Aslin, 2017; Laing \& Bergelson, under review). Four
recordings are analyzed for each infant: an audio- and video-recording
at 6 and 7 months, each on different
days\footnote{One video is missing due to technical error.}. See Table
\ref{tab:recording-ages-table}. Recordings that parents approved for
sharing with researchers are on Databrary.

\subsection{Video-Recordings}\label{video-recordings}

Researchers visited infants' homes each month to video-record a typical
hour of infants' lives. Infants were outfitted with a hat or headband
affixed with two small Looxcie cameras (22g each). One camera was
oriented slightly down and the other slightly up, to best capture
infant's visual field (verified via Bluetooth with an iPad/iPhone during
setup). A standard camcorder on a tripod (Panasonic HC-V100 or Sony
HDR-CX240) was positioned in the corner, which parents were asked to
move if they changed rooms. After set-up, experimenters left for one
hour.

\subsection{Audio-Recordings}\label{audio-recordings}

Audio-recordings captured up to 16 hours of infants' input. Parents were
given small audio-recorders (\textless{}60g) called LENAs (LENA
Foundation, Boulder, CO), along with vests with LENA-sized chest
pockets. Parents were asked to put the vest and recorder on babies from
when they awoke to when they went to bed (excepting naps and baths).
Parents were permitted to pause the recorder anytime but were asked to
minimize such pauses.

\subsection{Data Processing}\label{data-processing}

Details of the entire data-processing pipeline are on OSF
(\url{https://osf.io/cxwyz/wiki/home/}). Videos were processed using
Vegas and in-house scripts. Footage was aligned in a single,
multi-camera view before manual language annotation in Datavyu.
Audio-recordings were initially processed by LENA proprietary software,
which segments and diarizes each audio file; this output was then
converted to CLAN format (MacWhinney \& Wagner, 2010). After in-house
scripts marked long periods of silence (e.g.~naptimes) in CLAN, these
files were used for manual language annotation.

Modally, videos were an hour (62min., \emph{M}=60.79min., SD=6.31,
R=27.9--74.9min.), and audio-recordings were 16hrs. (960min.,
\emph{M}=858.41min., SD=119.41, R=635--960min.), LENA's maximum
capacity. Removing the long silences from audio-recording left
\textasciitilde{}10hrs. of audio (Mode=654min., \emph{M}=603min.,
SD=106.8, R=385.2--951min.), in line with established wakeful daytime
norms for 6--8-month-olds in the U.S. (Mindell, Sadeh, Wiegand, How, \&
Goh, 2010). All infants were awake for video-recording except one, whose
video annotation ended at sleep onset.

\subsection{Language Annotation}\label{language-annotation}

Trained researchers annotated each recording. This entailed demarcating
each concrete noun directed to or said loudly and clearly near the child
(e.g.~at adjacent siblings), but not distant language (e.g.~background
television).\enquote{Object words} were operationalized as concrete,
imageable nouns (e.g.~shoe, arm). Each annotation noted the noun and
lemma (e.g.~teethies, tooth), along with \emph{utterance-type},
\emph{object presence}, and \emph{talker}. \emph{Utterance-type}
classified each noun's utterance as declarative, question, imperative,
reading, singing, short-phrase, or unclear. (Short-phrases included
isolated words and \textless{}3-word noun-phrases, e.g. \enquote{the red
ball} or \enquote{kitty's paw}.) \emph{Object-presence} coded whether
objects were present and attended to (yes/no) based on linguistic
context (e.g. \enquote{here's your spoon!} was scored \enquote{yes});
for videos visual context was also used. Lastly, \emph{talker} tagged
live interlocutors and electronics, checked by staff highly familiar
with each family. We assessed intercoder reliability on a random
contiguous 10\% of annotations in each recording for the two categorical
variables (utterance-type and object-presence). Reliability was moderate
to strong (utterance-type: 87\% agreement, Cohen's \(\kappa\)=0.81;
object-presence: 83\% agreement, Cohen's \(\kappa\)=0.65).

\section{Results}\label{results}

\subsection{Analysis Plan}\label{analysis-plan}

Based on the coding scheme above, we derived 12 measures from each
recordings for each child (n=44), recording-type (audio, video), and
month (6, 7). See Table \ref{tab:measures-tab}. We averaged across
months to increase precision, and because we lacked
theoretically-motivated reasons to predict cross-month differences
(i.e.~no developmental or linguistic milestones are typically achieved
at 6--7mo.) Unfortunately, multi-level models were not vible due to
highly skewed residuals (by Shapiro-Wilk Test), even when
log-transformed, limiting cross-measure interpretation. Instead we
report a simple set of nonparametric analyses, conducted in R. The code
that rendered this manuscript is on Github, to be shared upon
publication.\footnote{Please contact corresponding author for access before publication.}

For all recording-type comparisons, we look at whether our measures
\emph{differed} significantly (by two-tailed, paired Wilcoxon Test) and
\emph{correlated} significantly (by Kendall Rank Correlation) across the
given groups. This approach lets us compare, e.g., whether the
time-normalized count of declarative nouns is indistinguishable in our
audio- and video-recordings, independently of whether these values are
correlated. We applied Holm's \emph{p}-value adjustment for multiple
comparisons (Holm, 1979) for each set of Wilcoxon tests and Kendall
Correlations.

\subsection{Count Measure Analysis}\label{count-measure-analysis}

To examine how the hour-long video data scale to day-length data
descriptively, we first divided the 12 count measures from the videos by
those from the audio-recordings for each child, to derive
\enquote{video-fraction} scores (video/audio). We opted for
video-fractions (rather than audio/video) to minimize undefined values
(e.g.~34\% of children heard no nouns in reading utterances in their
video-recordings; see Table \ref{tab:propna-missing-tables}. This
analysis showed that the video-recordings were 0.07 of the length of
audio-recordings, or 0.10 with audio-recording silences removed.
However, rather than a concomitant 10-fold decrease in our count
measures (as would be expected if videos captured a
\enquote{representative} hour of the day), the fractions averaged to
0.31; see Table \ref{tab:normtable}. Thus, by and large, videos had a
denser concentration of nouns across measures than did the
audio-recordings. See Figure \ref{fig:gr-derived-counts-67-diff} for raw
count data for each measure.

We next normed our counts by recording minutes. E.g., if an infant heard
500 noun-tokens in 800 non-silent audio-recording minutes, and 200 in 60
video minutes, this was normed to .62 and 3.3 noun-tokens/minute,
respectively; zero values were
retained.\footnote{One infant's data was excluded from 'father' measures; this infant had no father at home.}

We first looked at correlations across recording types, and find that
10/12 metrics correlated in audio vs.~video data; nouns per minute heard
from fathers and in singing did not. The size of the correlations
(i.e.~Kendall's \(\tau\)) was moderate (excluding the two
non-significant metrics, \emph{M}=0.44, 0.27-0.57, all
adjusted-\emph{p}\textless{}.05). See Table \ref{tab:normtable} and
Figure \ref{fig:gr-derived-normcounts-corr}.

We next compared the rates of each measure in three ways. First, we used
the normed data, looking at counts per minute. With the normed data,
11/12 metrics occurred at significantly lower rates in audio-recordings
than video-recordings (all adjusted-\emph{p}\textless{}.05). The
remaining metric, nouns from fathers, was statistically
indistinguishable across recording types
(adjusted-\emph{p}\textgreater{}.05). Thus, overall, per unit time,
infants heard less noun input across our metrics of quantity, talker,
utterance-type and object presence in audio-recordings than in videos
(see Figure \ref{fig:gr-derived-counts-67-diff} and
\ref{fig:gr-derived-normcounts-diff}).

Next, we compared two different hour-long subsets from the daylong audio
recording for comparison with the video-recorded hour, collectively
referred to as \enquote{peak} audio times. The \emph{top} hour was the
hour in which infants heard the most nouns. Complementarily, under the
logic that parents scheduled video-recordings to optimize infant
altertness, we extracted that \emph{same} hour in the daylong audio,
i.e.~if the video recording visit was scheduled from 2-3pm, we used
2-3pm from that child's daylong audio recording that month. Our 12
measures were then computed in both the \emph{top} and \emph{same} audio
hours. These hours only overlapped in 15/88 recordings
(17\%).\footnote{In 3 cases, the video-recording time (i.e. 'same' time) preceded the beginning of the daylong audio-recording (by 5, 30, or 90 minutes); in those cases the first hour of the recording was used. This created two further cases of 'top' and 'same' overlap.}

The results in video and same audio hours were indistinguishable for
8/12 measures; the remaining 4 occurred at significantly \emph{higher}
rates in the same audio hour (all adjusted-\emph{p}\textless{}.05): noun
types, nouns from fathers, and nouns in declaratives. Similarly, 7/12
occurred at significantly higher rates in top audio hour than in videos
(all adjusted-\emph{p}\textless{}.05); these included those from the
same audio comparison along with noun tokens, nouns in imperatives and
nouns in short phrases. Taken together, the videos presented a somewhat
different language input profile than the peak audio hours of the day:
videos featured less input for some quantity, talker, and utterance-type
measures, but were statistically indistinguishable in object presence,
input from mothers, and input in other utterance-types. This same
qualitative pattern held when looking at the rate of \enquote{zero}
values for the peak audio hours, relative to videos and daylong
audio-recordings (see Table \ref{tab:propna-missing-tables}).

\subsection{Exploratory Analyses}\label{exploratory-analyses}

Lastly, we undertook two sets of highly exploratory analyses, at the
utterance and word level. The utterance-type analysis is based on the
unanticipated result that while declaratives and questions made up
\textgreater{}2/3 of the input for each recording-type, the videos
appeared to contain relatively more questions and fewer declaratives
(See Fig. \ref{fig:gr-derived-counts-67-diff} and
\ref{fig:gr-derived-normcounts-diff}). To test this statistically, we
converted the six utterance-type counts to proportions (e.g. \#
declarative nouns/total nouns). Wilcoxon tests of each utterance-type in
audio- vs.~video-recording (corrected for multiple comparisons) revealed
that indeed, declaratives and questions occurred at different rates
across recording-types, with audio-recordings containing relatively
fewer questions (\(M_{video}\)=0.26, \(M_{audio}\)=0.19,
\(M_{same\  audio}\)=0.21, \(M_{top\  audio}\)=0.17) and more
declaratives than videos (\(M_{video}\)=0.40, \(M_{audio}\)=0.50,
\(M_{same\  audio}\)=0.49, \(M_{top\ audio}\)=0.47; each video vs.~audio
comparison adjusted-\emph{p}\textless{}.05). No other proportional
utterance-type differences reached significance across recording-types
(all adjusted-\emph{p}\textgreater{}.05). See Figure
\ref{fig:gr-ut-count-collapsed}.

At the word level, we aimed to characterize whether audio- and
video-recordings captured similar nouns at similar relative frequencies
across words and families. Nouns' frequency distribution was Zipfian: of
the 5801 unique object words (3137 lemmas), only 2482 (960 lemmas)
occurred \textgreater{}1 time.

We examined the 100 most frequent nouns from audio- and video-recordings
(n=136 due to ties, n=68 excluding words that never occurred in one
recording-type). Frequency across recording-types correlated
significantly (Kendall's \(\tau\): 0.39, \emph{p}\textless{}.0001) even
with zero-frequency words included (Kendall's \(\tau\): 0.25,
\emph{p}\textless{}.0001; see Figure \ref{fig:top-100-logspace} and
\ref{fig:top100-corr-rectype}).\footnote{The same pattern held with video compared to peak audio hours.}

Finally, we analyzed the top ten nouns within videos, daylong-audios,
and both peak audio hours. Four of the top ten words in each time sample
overlapped (baby, book, mouth, toes), suggesting that extremely common
nouns are relatively well-conserved. Moreover, all but one word in the
top 10 was identical for all 3 audio-based time-slices, while 5 of the
top video words were unique to video recordings (see Figure
\ref{fig:top10noun-freq}).

The top 10 words within each time sample also varied in how common they
were across the 44 families: top words from daylong audio occurred in
96\% of families (\emph{M}=42.30(2.63); those in video-recordings were
heard by 70\% (\emph{M}=31(6.27)). Nouns in peak audio hours patterned
in between (top hour: 88\% (\emph{M}=38.70(2.83); same hour: 78\%,
\emph{M}=34.20(4.71))

Finally, the top audio words were \textasciitilde{}3 times as common as
the top video words (\(M_{audio}\)=761.80(114.75),
\(M_{video}\)=232.80(91.38)), further underscoring the higher density of
nouns in video-recordings. Peak audio hour words were again in between
video and daylong audio (\(M_{top\ audio}\)=286.90(37.94);
\(M_{same\ audio}\)=210.40(26.72)). Taken together, this exploratory
analysis suggests that daylong audio-recordings may render more stable
estimates of pervasively common words across families than do
video-recordings.

\section{Discussion}\label{discussion}

Our results can be distilled to three key findings. First, the density
of noun input in hourlong video recordings was more similar to peak
times in daylong audio recordings rather than to the day at large. Per
minute, infants heard \textasciitilde{}2--4x more noun input across
quantity, speaker, utterance-type, and object-presence measures when
video-recorded for an hour versus audio-recorded for a day. Second,
while our metrics generally correlated across recording-types and many
gross patterns were conserved across them, audio- and video-recordings
differed in the relative rates of the top utterance types. That is,
videos featured more questions and fewer declaratives than
audio-recordings did. Finally, while the highest frequency words across
recording types largely overlapped and correlated, top words from the
daylong audio-recording appear to better represent the noun input across
families.

\subsection{Noun Quantity and Lexical
Diversity}\label{noun-quantity-and-lexical-diversity}

Overall, the pattern across recording-types primarily suggests a
difference in volubility, since by-and-large measures both correlated
and differed quantitatively by recording-type. As Suskind et al.(2013)
noted regarding interventions, daylong audio-recordings likely provide
more realistic counterparts to \enquote{best behavior} hourlong videos.
We add that shorter video-recording itself may influence volubility,
resulting in samples more akin to the high points in the natural ebb and
flow over the day.

Indeed, families likely found it simply easier to behave freely with
infants in special vests than with cameras on their heads. Our finding
that both \enquote{hat} and \enquote{camera} were top-10 video words
supports this idea; no analogous nouns (e.g.~vest, recorder) topped the
frequency rankings in audio-recordings (see Figures
\ref{fig:top-100-logspace} and \ref{fig:top10noun-freq}). Anecdotally,
while infants often required coaxing to wear the video-recording gear,
no analagous issues emerged for audio-recording.

Given that we held family and age constant, we expected many
similarities across recording-types; nevertheless, differences also
emerged. Indeed, the quantity metrics provide a conceptual replication
and extension of Tamis-LeMonda et al. (2017). Despite numerous
methodological variations (recording types and lengths, experimenter
presence, age, word-class analyzed), both studies found that parent talk
per unit time was significantly higher in shorter recordings on average,
but lower than the \emph{highest} portion of the longer recordings. This
general pattern appears robust across our sampling methods. Taken
together, this suggests that shorter recordings elicit denser, though
not maximal caregiver talk compared to what infants' typically
experience.

For certain research questions, such quantity differences may not
matter, e.g.~for studies examining \emph{relative} word rates or object
interactions in concentrated in-lab exposures. In contrast, research
quantifying language input across populations with varying demographic,
social, and cultural properties may need to be particularly sensitive to
cross-sample comparison (cf. Bergelson et al., under review; Cristia,
Dupoux, Gurven, \& Stieglitz, 2017; Shneidman \& Goldin-Meadow, 2012).

\subsection{Object Presence}\label{object-presence}

Rates of object presence were higher in videos than daylong
audio-recordings, but equivalent between videos and peak audio times.
Given that object presence correlated across recording-types within
children (0.40), our interpretation is that during higher talk volume
times (i.e.~video recordings and peak audio hours), nouns did occur with
more object presence (i.e.~infants mostly stayed in 1--2 rooms,
interacting with what was at hand). However, given that object presence
was coded based on linguistic context and when available, visual
context, it's possible that indistinguishable object presence across
video and peak audio is due to a combination of noise and systematic
bias in coding object presence without visual context. Given that object
presence and the related ideas of referential transparency and
contingent talk have been linked with early language development based
on both audio-only and video-recordings (Bergelson \& Aslin, 2017;
Cartmill et al., 2013; McGillion, Pine, Herbert, \& Matthews, 2017;
Yurovsky et al., 2013), we find this latter possibility somewhat
unlikely. Indeed, a better understanding of what elicits contingent,
referentially-transparent caretaker talk may be a fruitful avenue for
further work.

\subsection{Talker Variability}\label{talker-variability}

Infants heard nouns from more talkers per minute in videos than in
daylong audio-recordings. In contrast, infants heard roughly double the
speakers over the course of a day as they heard in one video-recorded
hour, and significantly more talkers during peak audio times than during
videos (see Fig. \ref{fig:gr-derived-counts-67-diff}).

Notably, while we considered noun input from all sources, 65.80\% of
infants' input came from mothers. Here, peak audio and video input from
mothers was equivalent, though in comparison with daylong audio, there
again were more nouns per minute from mothers in videos. In contrast,
input from fathers was the only measure that did not vary in videos
vs.~daylong audio-recordings in the normalized count data. However, in
the peak audio hours, there were more nouns from fathers than in the
videos. Relatedly, \textgreater{}50\% of videos captured \emph{no} input
from fathers. We believe this is because video-recording took place
during weekday business hours, when fathers in this sample were largely
at work, while audio-recordings spanned work-hours and days. Given that
fathers and mothers contribute differentially to early language
development (Pancsofar \& Vernon-Feagans, 2006), this is a clear example
of a consequence of methodological choices: to better understand
parents' input, considering work-schedules is critical. Put otherwise:
home-recordings scheduled at the researcher and primary caretaker's
convenience will likely undersample other caretakers.

The present results suggest that while infants hear most of their input
from their mothers, they also hear several other speakers during high
talk-volume times. Such data in turn feed infants' word-form
representations. Indeed, recent lab studies have found that at the same
age tested here, infants looked equivalently to named images when words
were produced by a new person or their mother (Bergelson \& Swingley,
2017), suggesting some degree of cross-talker normalization is in place
around 6 months. In contrast, 14-month-olds' learning of
similar-sounding words improves after training with tokens from multiple
speakers (Rost \& McMurray, 2010), suggesting that even small amounts of
talker variability aids new learning. This dovetails nicely with recent
work showing that talker variability differentally influences certain
phonetic discriminations (Bergmann et al., 2016). While a wide range of
talker and token distributionss surely result in appropriately
language-specific phonetic categories, we suggest that learning models
incorporating a large dose of input from a single talker alongside
smaller doses of input from 3+ other talkers may help inform word-form
knowledge in infants similar to those tested here.

\subsection{Utterance-Types}\label{utterance-types}

Per unit time, we found more nouns in every utterance-type in videos
than in daylong audio-recordings. In particular, we did not anticipate
differences in declaratives and questions. Indeed, while these
utterance-types universally made up the majority of noun input, videos
had relatively more questions and fewer declaratives. This is a key
instance where methodological choices may influence language acquisition
theories: base-rates of questions taken from videos would inflate
estimates of auxiliary verbs in the input. Notably, previous work has
varied in whether links between questions in the input and children's
early productions emerged, with developmental level invoked to explain
cross-study differences (Barnes, Gutfreund, Satterly, \& Wells, 1983;
cf. Huttenlocher, Vasilyeva, Cymerman, \& Levine, 2002). We add the
possibility that recording-type too may contribute to the base-rates of
questions in the input, even with age and recording length kept
constant.

\subsection{Top Words}\label{top-words}

Our third key finding concerned noun frequency and commonality across
families. We found that top words in the daylong audio-recordings were
heard by \(\geq84\%\) of families; only 1/10 top video words
(\enquote{hat}) was this common, a clear vestige of our recording
equipment (see Figure \ref{fig:top10noun-freq}). This result may be
meaningful in several ways. First, our analysis suggests that the input
would seem far more heterogeneous across children based on hour-long
video-recordings than it really is. Second, word frequency and
prevalence are often used to select stimuli for in-lab study; relying on
estimates from shorter, less representative recordings may stymie the
words studied in the lab. Thus, understanding how cross-family
noun-input stability scales with recording-length and type may prove
critical for future research; the word-level results above are an
initial exploration in understanding this dimension of naturalistic
observational data.

\subsection{Limitations and
Conclusions}\label{limitations-and-conclusions}

Given the technical limitation on battery life for small
video-recorders, we cannot conclusively separate the effects of modality
and length. That is, had we recorded daylong videos, we may have
obtained equivalent results across recording-types. Indeed, our peak
audio analyses provided some evidence that videos are more akin to
particularly language-saturated parts of infants' experience. However,
the peak audio comparisons are imperfect since these hours were not
bookended by researchers arriving and departing with pesky gear; further
comparisons await technological progress. Importantly, we do not mean to
suggest that audio reigns supreme: for many language-relevant questions
concerning gaze, gesture, and visual perception, it is simply
insufficient.

A further limitation here is self-selection: many parents are unwilling
to invite home recordings. Relatedly, our participants do not reflect US
demographics (let alone those elsewhere), and should be extended to
other populations before conclusive generalizations about sampling
methodology can be made (cf. Bergelson et al., under review).

Understanding what infants learn from is a key part of understanding
what and how they learn at all. These are first steps in unpacking how
two different data collection approaches may influence conclusions about
early linguistic input, with a narrow focus on the initially dominant
lexical class of nouns. We find that even naturalistic observer-free
video-recordings appear to inflate language input relative to daylong
recordings, in ways that influence syntactic constructions,
word-specific experiences, talker-variability, and the sheer quantity
and diversity of nouns infants hear. Work from the preceding decades
suggests these factors matter for early learning. Yet without knowing
how sampling methods may constrain results, we necessarily limit
adequate models of language acquisition. The present work charts
datapoints within this largely underspecified space, probing the
robustness of linguistically-relevant measures across naturalistic
sampling methods of infants' everyday experiences.\newpage

\begin{table}[tbp]
\begin{center}
\begin{threeparttable}
\caption{\label{tab:recording-ages-table}Infant ages at home recordings and enrollment lab visit}
\begin{tabular}{llll}
\toprule
Month & \multicolumn{1}{c}{Video Recordings} & \multicolumn{1}{c}{Audio Recordings} & \multicolumn{1}{c}{In-lab visits}\\
\midrule
6 & M=6;4, SD=3.2 days & M=6;7, SD=3.9 days & M=6;2, SD=3.7 days\\
7 & M=7;2, SD=2.3 days & M=7;5, SD=3.3 days & NA\\
\bottomrule
\end{tabular}
\end{threeparttable}
\end{center}
\end{table}

\pagebreak

\begin{table}[tbp]
\begin{center}
\begin{threeparttable}
\caption{\label{tab:measures-tab}Count measures (n=12), by Measure-Type}
\small{
\begin{tabular}{ll}
\toprule
Measure & \multicolumn{1}{c}{Derived Count}\\
\midrule
Quantity & Noun tokens, Noun types\\
Speaker & Nouns from Mother, Nouns from Father, Unique Speakers\\
Utterance Type & Nouns in Declaratives, Imperatives, Questions, Short-Phrases, Reading, or Singing\\
Object Presence & Nouns said when the referent was present and attended to\\
\bottomrule
\end{tabular}
}
\end{threeparttable}
\end{center}
\end{table}

\pagebreak

\begin{table}[tbp]
\begin{center}
\begin{threeparttable}
\caption{\label{tab:propna-missing-tables}Proportion of infants with no recorded nouns for the listed measures, by sample}
\small{
\begin{tabular}{llllll}
\toprule
Time Sample & \multicolumn{1}{c}{Fathers} & \multicolumn{1}{c}{Mothers} & \multicolumn{1}{c}{Reading} & \multicolumn{1}{c}{Singing} & \multicolumn{1}{c}{Imperatives}\\
\midrule
A & NA & NA & 0.16 & 0.02 & NA\\
SameA & 0.27 & NA & 0.43 & 0.11 & NA\\
TopA & 0.18 & 0.02 & 0.27 & 0.07 & NA\\
V & 0.51 & 0.09 & 0.34 & 0.11 & 0.02\\
\bottomrule
\addlinespace
\end{tabular}
}
\begin{tablenotes}[para]
\textit{Note.} V=video, A=daylong audio, Top=top hour of A, Same = Video-hour of A. All infants heard nouns for all other measures (see Table 2).
\end{tablenotes}
\end{threeparttable}
\end{center}
\end{table}

\pagebreak

\begin{table}[tbp]
\begin{center}
\begin{threeparttable}
\caption{\label{tab:normtable}Video/Audio Count Measures, normed by minutes in recording (column 2) and divided without norming (column 3)}
\small{
\begin{tabular}{lll}
\toprule
Measure & \multicolumn{1}{c}{Inflation (normed)} & \multicolumn{1}{c}{Video-fraction Mean(SD)}\\
\midrule
Minutes & NA & 0.07 (0.01)\\
Awake minutes & NA & 0.1 (0.02)\\
Types & 3.00 & 0.31 (0.13)\\
Tokens & 2.30 & 0.25 (0.15)\\
Speakers & 3.90 & 0.44 (0.2)\\
Mother & 3.00 & 0.32 (0.22)\\
Father & 1.10 & 0.13 (0.26)\\
Declaratives & 1.90 & 0.19 (0.09)\\
Questions & 3.10 & 0.33 (0.16)\\
Imperatives & 2.60 & 0.27 (0.23)\\
Singing & 2.30 & 0.65 (1.46)\\
Reading & 2.90 & 1.02 (2.76)\\
Short phrases & 2.50 & 0.3 (0.25)\\
Object presence & 2.90 & 0.34 (0.28)\\
\bottomrule
\addlinespace
\end{tabular}
}
\begin{tablenotes}[para]
\textit{Note.} If videos contained equivalent quantities of nouns, Inflation values would be 1, and Video-fractions would be .1
\end{tablenotes}
\end{threeparttable}
\end{center}
\end{table}

\begin{figure}
\centering
\includegraphics{sixseven_simplepapaja_files/figure-latex/gr-derived-counts-67-diff-1.pdf}
\caption{\label{fig:gr-derived-counts-67-diff}Noun count measures across
audio-recordings and videos. Top row depicts daylong audio data; bottom
row shows the 3 hour-long annotations: \enquote{same} and \enquote{top}
are the two peak audio times, and \enquote{video} indicates the video
data. Upper panel labels indicate annotated sample length (day or hour);
the bottom panel labels reflects measure type (op = object presence; utt
= utterance-type, quant = quantity, Nspeakers = number of speakers).
Bars (left to right) appear in legend order (top to bottom) in both
color (count measures) and opacity (time sample: day, top-hour,
same-hour, or video).}
\end{figure}

\begin{figure}
\centering
\includegraphics{sixseven_simplepapaja_files/figure-latex/gr-derived-normcounts-diff-1.pdf}
\caption{\label{fig:gr-derived-normcounts-diff}Noun count measures
normalized by recording length, for audio-recordings (solid borders) and
videos (dashed borders). Normalized counts were calculated by dividing
raw counts (see Fig 1.) by non-silent recording minutes. op = object
presence; utt = utterance-type, quant = quantity, Nspeakers = number of
speakers. Bars (left to right) appear in legend order (top to bottom).
All measures differed significantly across recording-types except nouns
from fathers.}
\end{figure}

\begin{figure}
\centering
\includegraphics{sixseven_simplepapaja_files/figure-latex/gr-derived-normcounts-corr-1.pdf}
\caption{\label{fig:gr-derived-normcounts-corr}Normalized count correlations
between audio- vs.~video-recordings. Each point indicates nouns per
minute of recording for each child, averaged across months 6 and 7, for
each measure. Point-shape indicates measure type. Robust linear
correlations are plotted for visualization only; non-parametric
correlations (Kendall) were computed for analysis, showing that all
correlations were significant except nouns from fathers and in singing.}
\end{figure}

\begin{figure}
\centering
\includegraphics{sixseven_simplepapaja_files/figure-latex/gr-ut-count-collapsed-1.pdf}
\caption{\label{fig:gr-ut-count-collapsed}Utterance-type proportions across
audio-recordings (daylong, \enquote{same} hour and \enquote{top} hour)
and videos (indicated by line-type). Utterance-types are in legend order
top to bottom. Videos contained a significantly more questions and fewer
declaratives than the audio-recording time samples.}
\end{figure}

\begin{figure}
\centering
\includegraphics{sixseven_simplepapaja_files/figure-latex/top-100-logspace-1.pdf}
\caption{\label{fig:top-100-logspace}Log-scaled counts of the top 100 words
in audio- and video-recordings. Each node represents the averaged count,
across all participants in both months, of each noun (0.1 was added
before taking logs to include 0 counts.) Words in blue occurred 0 times
in one recording type; words in pink were attested in both recording
types. Nodes are jittered for visual clarity, with grey lines indicating
node location on axes.}
\end{figure}

\begin{figure}
\centering
\includegraphics{sixseven_simplepapaja_files/figure-latex/top100-corr-rectype-1.pdf}
\caption{\label{fig:top100-corr-rectype}Correlations of the frequencies of
the top 100 words in audio- vs.~video-recordings. Each node represents
one word averaged across all participants in both months.}
\end{figure}

\begin{figure}
\includegraphics{sixseven_simplepapaja_files/figure-latex/top10noun-freq-1} \caption{Top 10 words by recording type and time sample. Each node represents the frequency count of each top audio or video word over both months (x-axis) and the number of families where that word was said (out of 44) across months (y-axis).}\label{fig:top10noun-freq}
\end{figure}

\setlength{\parindent}{-0.5in} \setlength{\leftskip}{0.5in}

\hypertarget{refs}{}
\hypertarget{ref-barnes1983characteristics}{}
Barnes, S., Gutfreund, M., Satterly, D., \& Wells, G. (1983).
Characteristics of adult speech which predict children's language
development. \emph{Journal of Child Language}, \emph{10}(1), 65?84.
doi:\href{https://doi.org/10.1017/S0305000900005146}{10.1017/S0305000900005146}

\hypertarget{ref-bergelson2017nature}{}
Bergelson, E., \& Aslin, R. N. (2017). Nature and origins of the lexicon
in 6-mo-olds. \emph{Proceedings of the National Academy of Sciences},
\emph{114}(49), 12916--12921.

\hypertarget{ref-bergelson2013acquisition}{}
Bergelson, E., \& Swingley, D. (2013). The acquisition of abstract words
by young infants. \emph{Cognition}, \emph{127}(3), 391--397.

\hypertarget{ref-bergelson2017young}{}
Bergelson, E., \& Swingley, D. (2017). Young infants' word comprehension
given an unfamiliar talker or altered pronunciations. \emph{Child
Development}.

\hypertarget{ref-bergelsonunderreview}{}
Bergelson, E., Casillas, M., Soderstrom, M., Seidl, A., Warlaumont, A.,
\& Amatuni, A. (under review). What do north american babies hear? A
large-scale cross-corpus analysis.

\hypertarget{ref-bergmann2016discriminability}{}
Bergmann, C., Cristia, A., \& Dupoux, E. (2016). Discriminability of
sound contrasts in the face of speaker variation quantified. In
\emph{Proceedings of the 38th annual meeting of the cognitive science
society} (Vol. 510).

\hypertarget{ref-brent2001role}{}
Brent, M. R., \& Siskind, J. M. (2001). The role of exposure to isolated
words in early vocabulary development. \emph{Cognition}, \emph{81}(2),
B33--B44.

\hypertarget{ref-cartmill2013quality}{}
Cartmill, E. A., Armstrong, B. F., Gleitman, L. R., Goldin-Meadow, S.,
Medina, T. N., \& Trueswell, J. C. (2013). Quality of early parent input
predicts child vocabulary 3 years later. \emph{Proceedings of the
National Academy of Sciences}, \emph{110}(28), 11278--11283.

\hypertarget{ref-cristia2017child}{}
Cristia, A., Dupoux, E., Gurven, M., \& Stieglitz, J. (2017).
Child-directed speech is infrequent in a forager-farmer population: A
time allocation study. \emph{Child Development}.

\hypertarget{ref-dale1996lexical}{}
Dale, P. S., \& Fenson, L. (1996). Lexical development norms for young
children. \emph{Behavior Research Methods, Instruments, \& Computers},
\emph{28}(1), 125--127.

\hypertarget{ref-debaryshe1993joint}{}
DeBaryshe, B. D. (1993). Joint picture-book reading correlates of early
oral language skill. \emph{Journal of Child Language}, \emph{20}(2),
455--461.

\hypertarget{ref-dudley2009pathologizing}{}
Dudley-Marling, C., \& Lucas, K. (2009). Pathologizing the language and
culture of poor children. \emph{Language Arts}, \emph{86}(5), 362--370.

\hypertarget{ref-fernald2013ses}{}
Fernald, A., Marchman, V. A., \& Weisleder, A. (2013). SES differences
in language processing skill and vocabulary are evident at 18 months.
\emph{Developmental Science}, \emph{16}(2), 234--248.

\hypertarget{ref-hart1995meaningful}{}
Hart, B., \& Risley, T. R. (1995). \emph{Meaningful differences in the
everyday experience of young american children.} Paul H Brookes
Publishing.

\hypertarget{ref-hoff2002children}{}
Hoff, E., \& Naigles, L. (2002). How children use input to acquire a
lexicon. \emph{Child Development}, \emph{73}(2), 418--433.

\hypertarget{ref-holm1979simple}{}
Holm, S. (1979). A simple sequentially rejective multiple test
procedure. \emph{Scandinavian Journal of Statistics}, \emph{6}(2),
65--70. Retrieved from \url{http://www.jstor.org/stable/4615733}

\hypertarget{ref-huttenlocher2002language}{}
Huttenlocher, J., Vasilyeva, M., Cymerman, E., \& Levine, S. (2002).
Language input and child syntax. \emph{Cognitive Psychology},
\emph{45}(3), 337--374.

\hypertarget{ref-Laing_Bergelson_17}{}
Laing, C., \& Bergelson, E. (under review). The effect of mothers' work
schedule on 17-month-olds' productive vocabulary.

\hypertarget{ref-lidz2003infants}{}
Lidz, J., Waxman, S., \& Freedman, J. (2003). What infants know about
syntax but couldn't have learned: Experimental evidence for syntactic
structure at 18 months. \emph{Cognition}, \emph{89}(3), 295--303.

\hypertarget{ref-macwhinney2001emergentist}{}
MacWhinney, B. (2001). Emergentist approaches to language.
\emph{TYPOLOGICAL STUDIES IN LANGUAGE}, \emph{45}, 449--470.

\hypertarget{ref-macwhinney2010transcribing}{}
MacWhinney, B., \& Wagner, J. (2010). Transcribing, searching and data
sharing: The clan software and the talkbank data repository.
\emph{Gesprachsforschung: Online-Zeitschrift Zur Verbalen Interaktion},
\emph{11}, 154.

\hypertarget{ref-mcgillion2017randomised}{}
McGillion, M., Pine, J. M., Herbert, J. S., \& Matthews, D. (2017). A
randomised controlled trial to test the effect of promoting caregiver
contingent talk on language development in infants from diverse
socioeconomic status backgrounds. \emph{Journal of Child Psychology and
Psychiatry}.

\hypertarget{ref-michaels2013commentary}{}
Michaels, S. (2013). Commentary: Déjà vu all over again: What's wrong
with hart \& risley and a`` linguistic deficit'' framework in early
childhood education? \emph{Learning Landscapes}, \emph{7}(1), 23--41.

\hypertarget{ref-mindell2010cross}{}
Mindell, J. A., Sadeh, A., Wiegand, B., How, T. H., \& Goh, D. Y.
(2010). Cross-cultural differences in infant and toddler sleep.
\emph{Sleep Medicine}, \emph{11}(3), 274--280.

\hypertarget{ref-noble2005neurocognitive}{}
Noble, K. G., Norman, M. F., \& Farah, M. J. (2005). Neurocognitive
correlates of socioeconomic status in kindergarten children.
\emph{Developmental Science}, \emph{8}(1), 74--87.

\hypertarget{ref-oller2010automated}{}
Oller, D. K., Niyogi, P., Gray, S., Richards, J., Gilkerson, J., Xu, D.,
\ldots{} Warren, S. (2010). Automated vocal analysis of naturalistic
recordings from children with autism, language delay, and typical
development. \emph{Proceedings of the National Academy of Sciences},
\emph{107}(30), 13354--13359.

\hypertarget{ref-pancsofar2006mother}{}
Pancsofar, N., \& Vernon-Feagans, L. (2006). Mother and father language
input to young children: Contributions to later language development.
\emph{Journal of Applied Developmental Psychology}, \emph{27}(6),
571--587.

\hypertarget{ref-rost2010finding}{}
Rost, G. C., \& McMurray, B. (2010). Finding the signal by adding noise:
The role of noncontrastive phonetic variability in early word learning.
\emph{Infancy}, \emph{15}(6), 608--635.

\hypertarget{ref-roy2015predicting}{}
Roy, B. C., Frank, M. C., DeCamp, P., Miller, M., \& Roy, D. (2015).
Predicting the birth of a spoken word. \emph{Proceedings of the National
Academy of Sciences}, \emph{112}(41), 12663--12668.

\hypertarget{ref-shneidman2012language}{}
Shneidman, L., \& Goldin-Meadow, S. (2012). Language input and
acquisition in a mayan village: How important is directed speech?,
\emph{15}, 659--73.

\hypertarget{ref-suskind2013exploratory}{}
Suskind, D., Leffel, K. R., Hernandez, M. W., Sapolich, S. G., Suskind,
E., Kirkham, E., \& Meehan, P. (2013). An exploratory study of
``quantitative linguistic feedback'': Effect of lena feedback on adult
language production. \emph{Communication Disorders Quarterly},
\emph{34}(4), 199--209.
doi:\href{https://doi.org/10.1177/1525740112473146}{10.1177/1525740112473146}

\hypertarget{ref-taine1876note}{}
Taine, H. (1876). Note sur l'acquisition du langage chez les enfants et
dans l'espèce humaine. \emph{Revue Philosophique de La France et de
L'Etranger}, 5--23.

\hypertarget{ref-tamis2017power}{}
Tamis-LeMonda, C. S., Kuchirko, Y., Luo, R., Escobar, K., \& Bornstein,
M. H. (2017). Power in methods: Language to infants in structured and
naturalistic contexts. \emph{Developmental Science}.

\hypertarget{ref-tomasello2000young}{}
Tomasello, M. (2000). Do young children have adult syntactic competence?
\emph{Cognition}, \emph{74}(3), 209--253.

\hypertarget{ref-vandam2016homebank}{}
VanDam, M., Warlaumont, A. S., Bergelson, E., Cristia, A., Soderstrom,
M., De Palma, P., \& MacWhinney, B. (2016). HomeBank: An online
repository of daylong child-centered audio recordings. In \emph{Seminars
in speech and language} (Vol. 37, pp. 128--142). Thieme Medical
Publishers.

\hypertarget{ref-weisleder2013talking}{}
Weisleder, A., \& Fernald, A. (2013). Talking to children matters: Early
language experience strengthens processing and builds vocabulary.
\emph{Psychological Science}, \emph{24}(11), 2143--2152.

\hypertarget{ref-williams1937analytical}{}
Williams, H. M. (1937). An analytical study of language achievement in
preschool children. \emph{University of Iowa Studies in Child Welfare},
\emph{13}, 9--18.

\hypertarget{ref-yurovsky2013statistical}{}
Yurovsky, D., Smith, L. B., \& Yu, C. (2013). Statistical word learning
at scale: The baby's view is better. \emph{Developmental Science},
\emph{16}(6), 959--966.






\end{document}
