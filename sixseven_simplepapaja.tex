\documentclass[floatsintext,man]{apa6}

\usepackage{amssymb,amsmath}
\usepackage{ifxetex,ifluatex}
\usepackage{fixltx2e} % provides \textsubscript
\ifnum 0\ifxetex 1\fi\ifluatex 1\fi=0 % if pdftex
  \usepackage[T1]{fontenc}
  \usepackage[utf8]{inputenc}
\else % if luatex or xelatex
  \ifxetex
    \usepackage{mathspec}
    \usepackage{xltxtra,xunicode}
  \else
    \usepackage{fontspec}
  \fi
  \defaultfontfeatures{Mapping=tex-text,Scale=MatchLowercase}
  \newcommand{\euro}{€}
\fi
% use upquote if available, for straight quotes in verbatim environments
\IfFileExists{upquote.sty}{\usepackage{upquote}}{}
% use microtype if available
\IfFileExists{microtype.sty}{\usepackage{microtype}}{}

% Table formatting
\usepackage{longtable, booktabs}
\usepackage{lscape}
% \usepackage[counterclockwise]{rotating}   % Landscape page setup for large tables
\usepackage{multirow}		% Table styling
\usepackage{tabularx}		% Control Column width
\usepackage[flushleft]{threeparttable}	% Allows for three part tables with a specified notes section
\usepackage{threeparttablex}            % Lets threeparttable work with longtable

% Create new environments so endfloat can handle them
% \newenvironment{ltable}
%   {\begin{landscape}\begin{center}\begin{threeparttable}}
%   {\end{threeparttable}\end{center}\end{landscape}}

\newenvironment{lltable}
  {\begin{landscape}\begin{center}\begin{ThreePartTable}}
  {\end{ThreePartTable}\end{center}\end{landscape}}




% The following enables adjusting longtable caption width to table width
% Solution found at http://golatex.de/longtable-mit-caption-so-breit-wie-die-tabelle-t15767.html
\makeatletter
\newcommand\LastLTentrywidth{1em}
\newlength\longtablewidth
\setlength{\longtablewidth}{1in}
\newcommand\getlongtablewidth{%
 \begingroup
  \ifcsname LT@\roman{LT@tables}\endcsname
  \global\longtablewidth=0pt
  \renewcommand\LT@entry[2]{\global\advance\longtablewidth by ##2\relax\gdef\LastLTentrywidth{##2}}%
  \@nameuse{LT@\roman{LT@tables}}%
  \fi
\endgroup}


  \usepackage{graphicx}
  \makeatletter
  \def\maxwidth{\ifdim\Gin@nat@width>\linewidth\linewidth\else\Gin@nat@width\fi}
  \def\maxheight{\ifdim\Gin@nat@height>\textheight\textheight\else\Gin@nat@height\fi}
  \makeatother
  % Scale images if necessary, so that they will not overflow the page
  % margins by default, and it is still possible to overwrite the defaults
  % using explicit options in \includegraphics[width, height, ...]{}
  \setkeys{Gin}{width=\maxwidth,height=\maxheight,keepaspectratio}
\ifxetex
  \usepackage[setpagesize=false, % page size defined by xetex
              unicode=false, % unicode breaks when used with xetex
              xetex]{hyperref}
\else
  \usepackage[unicode=true]{hyperref}
\fi
\hypersetup{breaklinks=true,
            pdfauthor={},
            pdftitle={Talk, You're On Camera! Or, Comparing Naturalistic Audio and Video Recordings of Infants},
            colorlinks=true,
            citecolor=blue,
            urlcolor=blue,
            linkcolor=black,
            pdfborder={0 0 0}}
\urlstyle{same}  % don't use monospace font for urls

\setlength{\parindent}{0pt}
%\setlength{\parskip}{0pt plus 0pt minus 0pt}

\setlength{\emergencystretch}{3em}  % prevent overfull lines


% Manuscript styling
\captionsetup{font=singlespacing,justification=justified}
\usepackage{csquotes}
\usepackage{upgreek}



\usepackage{tikz} % Variable definition to generate author note

% fix for \tightlist problem in pandoc 1.14
\providecommand{\tightlist}{%
  \setlength{\itemsep}{0pt}\setlength{\parskip}{0pt}}

% Essential manuscript parts
  \title{Talk, You're On Camera! Or, Comparing Naturalistic Audio and Video
Recordings of Infants}

  \shorttitle{Talk, You're On Camera!}


  \author{Elika Bergelson\textsuperscript{1}, Andrei Amatuni\textsuperscript{1}, Shannon Dailey\textsuperscript{1}, Sharath Koorathota\textsuperscript{3}, \& Shaelise Tor\textsuperscript{4}}

  % \def\affdep{{"", "", "", "", ""}}%
  % \def\affcity{{"", "", "", "", ""}}%

  \affiliation{
    \vspace{0.5cm}
          \textsuperscript{1} Duke University\\
          \textsuperscript{2} University of Rochester\\
          \textsuperscript{3} Columbia University Medical Center\\
          \textsuperscript{4} Syracuse University  }

  \authornote{
    Elika Bergelson, Psychology \& Neuroscience, Center for Cognitive
    Neuroscience, Center for Developmental Science, Duke University
    
    Andrei Amatuni, Psychology \& Neuroscience, Duke University
    
    Shannon Dailey, Psychology \& Neuroscience, Duke University
    
    Sharath Koorathota, Columbia University Medical Center
    
    Shaelise Tor, Marriage and Family Therapy, Syracuse University
    
    Correspondence concerning this article should be addressed to Elika
    Bergelson, 417 Chapel Drive, Box 90086. E-mail:
    \href{mailto:elika.bergelson@duke.edu}{\nolinkurl{elika.bergelson@duke.edu}}
  }


  \abstract{Measurements of infants' quotidian experiences provide critical
information about early development. However, the role of sampling
methods in providing this information is rarely examined. Here we
directly compare language input from hourlong videos and daylong
audio-recordings within the same group of 44 infants, at 6 and 7 months.
We find far denser noun input in video- than in audio-recordings, across
12 measures of language quantity and lexical diversity, talker
variability, utterance-type, and referential transparency. Although
audio-recordings captured \textasciitilde{}10x more awake-time than
videos, the noun input in them was only 2-5x greater. Most notably, per
unit time, videos featured more word-types and tokens, more questions
but fewer declaratives, and more talkers. In contrast,
\textgreater{}33\% of videos lacked certain noun input altogether,
e.g.~reading and fathers' speech. While we find moderate correlations
across recording-types, the most common audio-recording nouns were far
more consistant across families than top video-recording nouns. Thus,
hour-long videos and daylong audio-recordings provided fairly divergent
pictures of the input infants hear and learn from in their daily lives.
We suggest short video-recordings may inflate various language input
estimates, and should be used cautiously for extrapolation about common
words, talkers, utterance-types, and contexts at larger timescales. If
theories of language development are to be held accountable to `facts on
the ground' from observational data, greater care is needed to unpack
the ramifications of sampling methods of early language input.}
  




\usepackage{amsthm}
\newtheorem{theorem}{Theorem}
\newtheorem{lemma}{Lemma}
\theoremstyle{definition}
\newtheorem{definition}{Definition}
\newtheorem{corollary}{Corollary}
\newtheorem{proposition}{Proposition}
\theoremstyle{definition}
\newtheorem{example}{Example}
\theoremstyle{definition}
\newtheorem{exercise}{Exercise}
\theoremstyle{remark}
\newtheorem*{remark}{Remark}
\newtheorem*{solution}{Solution}
\begin{document}

\maketitle

\setcounter{secnumdepth}{0}



\section{Highlights}\label{highlights}

\begin{itemize}
\tightlist
\item
  We measured 44 infants' early noun input during free-form,
  infant-caregiver interactions, in hour-long videos and daylong
  audio-recordings, at 6 and 7 months
\item
  Across measures of quantity, utterance-type, object presence, and
  talker, nouns-per-minute were 2-5 times more frequent in videos than
  in audio-recordings
\item
  Videos had relatively more questions and fewer declaratives. The most
  frequent nouns across audio-recordings also occured in most families;
  this was not true for videos.
\item
  Methodological differences in naturalistic observations techniques
  have great influence on researchers' potential conclusions about
  infants' language input
\end{itemize}

Researchers have studied development by observing infants experiencing
their natural habitats for over a century (Taine, 1876; Williams, 1937).
Over the past 20-30 years, written records have been increasingly
supplemented with annotated audio- and video-recordings, which have
described the linguistic, social, and physical landscape in which
infants learn. Such data --often shared through repositories like
CHILDES and Databrary-- in turn provide a proxy for various
\enquote{input} measures in theories of psycho-social, motor, and in
particular, linguistic development (MacWhinney, 2001).

Furthermore, recent technological advances have made it feasible to
collect longer, denser, and higher-quality recordings of infants'
day-to-day lives, which aim to provide better approximations of infants'
input and early language abilities (Bergelson \& Aslin, 2017; Oller et
al., 2010; B. C. Roy, Frank, DeCamp, Miller, \& Roy, 2015; VanDam et
al., 2016; Weisleder \& Fernald, 2013, \emph{inter alia}). Such
naturalistic data seeks to reveal what infants actually learn from as
they make use of their biological endowments and environmental
resources.

While cutting edge technologies make collecting observational data ever
easier, this growing toolbox increases researchers' decision load, with
serious but underexplored side-effects. For instance, researchers must
decide on recording modalities (e.g.~audio, video, both), where, whom,
and how long to record, and whether to capture structured or
free-ranging interactions, with or without experimenters present. While
any path through such decision-trees may lead to equivalent results,
this is rarely tested directly. Problematically, this leads to research
with theoretical conclusions built on equivalency assumptions that go
unmeasured.

In recent work directly comparing observational sampling methods;
Tamis-LeMonda, Kuchirko, Luo, Escobar, and Bornstein (2017) analyzed
mother-infant behavior in 5-minute structured interactions, and 45
minutes of free play. Home sessions were video-recorded by an
experimenter and transcribed. The results showed that relative to free
play, in structured interactions infants generally experienced more
language both in word-quantity (i.e.~tokens) and word-variability
(i.e.~types) per minute. They also found that language quantity across
contexts correlated, and that the peak five-minutes of the naturalistic
interaction was similar to the 5-minute structured interaction. They
conclude that sampling must be matched with research-question,
cautioning that while brief samples may be appropriate for studying
individual differences, extrapolations about overall language input from
short samples must be made with care.

In contrast, work by Hart and Risley (1995) extrapolated extensively.
Based on 30 hours of data per family (collected one hour per month for
2.5 years), these researchers estimated that by age four, children
receiving public assistance (n=6) heard \textgreater{}30-million fewer
words than professional-class children (n=13). While their results
highlighting SES differences certainly merited (and received) follow-up
(e.g. Fernald, Marchman, \& Weisleder, 2013; Noble, Norman, \& Farah,
2005, \emph{inter alia}), they have also been criticized as an extreme
over-extrapolation (Dudley-Marling \& Lucas, 2009; Michaels, 2013).

Still other research analyzes base rates of certain linguistic
phenomena, to provide in-principle proof of what young children can
learn from their input (Brent \& Siskind, 2001; Lidz, Waxman, \&
Freedman, 2003; Tomasello, 2000). Here, the research question generally
dictated what was deemed appropriate sampling. Problematically, for most
exploratory work, \enquote{appropriate} sampling is hard to premeditate.
For instance, practically any length of adult speech, across
wide-ranging recording parameters will find function words (e.g.
\enquote{of}) at much higher rates than content words (e.g.
\enquote{fork}). But for questions concerning many aspects of infants'
language input, it is largely unknown how methodological choices may
bias our answers.

In the present work, we explore these issues, directly comparing
hour-long video-recordings and daylong audio-recordings in a single
sample of 44 infants, at 6 and 7 months, as part of a larger study on
early noun learning. We annotated concrete nouns (generally, objects,
foods, animals, or body-parts) said to infants, or said loudly and
clearly in their presence. We further annotated three properties
previously linked with early language learning: (1) utterance-type,
which provides syntactic and situational information (Brent \& Siskind,
2001; DeBaryshe, 1993; Hoff \& Naigles, 2002) (2) object presence
(i.e.~referential transparency) which clarifies whether the referent of
a spoken word is visually appreciable (Bergelson \& Aslin, 2017;
Bergelson \& Swingley, 2013; Cartmill et al., 2013; Yurovsky, Smith, \&
Yu, 2013), and (3) talker, which lets us quantify the range of speakers
infants hear (Bergmann, Cristia, \& Dupoux, 2016; Rost \& McMurray,
2010).

This design sets up two overarching questions. First, do features of the
noun input in one video-recorded hour predict these same quantities in
an entire audio-recorded day? Second, do input quantities differ once
time is standardized? If the noun input is equivalent and predictive
across recording-types, then researchers can freely vary their
observational data collection approach with impunity. If it is not,
understanding the biases of various methods is critical to ensuring our
learning theories consider the data quantity and variability available
to learners day-to-day.

Thus, our main goal was to compare language input young infants receive
across four key properties (word quantity/diversity, utterance-type,
object presence, and talker), as measured by hourlong videos and
(separate) full-day audio-recordings. This seemingly methodological
question has deep implications for developmental theory: we examine how
sampling approaches may alter conclusions about the linguistic input
that in turn drives early development.

\section{Methods}\label{methods}

\subsection{Participants}\label{participants}

Participants were recruited from an existing database of families from
local hospitals, or who heard about the BabyLab from friends, family,
and outreach. Forty-six participants enrolled; two dropped out in the
early stages of the project leaving 44 infants in the final sample. All
infants were full-term (40 ± 3 weeks), had no known vision or hearing
problems, and heard \(\geq 75\%\) spoken English in the home.
Participants were 95\% white; 75\% of mothers had a B.A. or higher. The
families were enrolled in a yearlong study that included monthly audio-
and video-recordings, as well as in-lab visits every other month. Here
we report on the home recording data from the first two timepoints (6
and 7 months) of this study, for which participants were compensated
\$10.\footnote{We include only these timepoints because no infants had begun producing words themselves (which changes the input for reasons orthogonal to those examined here); given the broader project aims, these timepoints alone had the entire daylong audio-recording annotated.}

\begin{table}[b]
\begin{center}
\begin{threeparttable}
\caption{\label{tab:recording-ages-table}Infant ages at recordings and lab visits}
\begin{tabular}{llll}
\toprule
Month & \multicolumn{1}{c}{Video Recordings} & \multicolumn{1}{c}{Audio Recordings} & \multicolumn{1}{c}{In-lab visits}\\
\midrule
6 months & M=6;4, SD=3.2 days & M=6;7, SD=3.9 days & M=6;2, SD=3.7 days\\
7 months & M=7;2, SD=2.3 days & M=7;5, SD=3.3 days & NA\\
\bottomrule
\end{tabular}
\end{threeparttable}
\end{center}
\end{table}

\subsection{Procedures}\label{procedures}

Participants gave consent at an initial lab visit for the larger study
through a process approved by the University of Rochester IRB.
Questionnaires about various aspects of the family's and infant's
background conducted during lab visits, not germane to the present
analysis, are reported elsewhere (Bergelson \& Aslin, 2017) \textbf{ADD
REF Laing and Bergelson, under review)}. Four recordings were collected
for each infant: an audio- and video-recording at six and seven months,
each on a different day. See Table \ref{tab:recording-ages-table}.

Audio-video release forms were given to parents and collected after the
audio and video recordings for the month were complete. Parents could
opt to share the data with other authorized researchers and/or to have
excerpts used for academic presentation. The released audio and video
files can be accessed by registered researchers on Databrary.

\subsection{Video-Recordings}\label{video-recordings}

Researchers visited infants' homes each month to video-record a typical
hour of infants' lives from their own perspective. To achieve this,
infants were outfitted with a hat or headband affixed with two small,
lightweight Looxcie cameras (22g each). One camera was oriented slightly
down and the other slightly up, to capture most of the infant's visual
field (verified by Bluetooth with an iPad/iPhone during setup). A
standard camcorder (Panasonic HC-V100 or Sony HDR-CX240) on a tripod was
set up in a location that could best capture the infant. Parents were
asked to move this camera with them if they changed rooms. After set-up,
experimenters left for one hour.

\subsection{Audio-Recordings}\label{audio-recordings}

Audio-recordings captured a full day (up to 16 hours) of infants'
language input. Parents were given vests with a small chest-pocket, and
LENAs (LENA Foundation, Boulder, CO), small audio-recorders
(\textless{}60g) that fit into the vest pocket. Parents were asked to
put the vest and recorder on babies from when they awoke to when they
went to bed (with the exceptions of naps and baths). Parents were
permitted to pause the recorder at any time but were asked to keep such
pauses minimal.

\subsection{Data Processing}\label{data-processing}

Details of our entire data processing pipeline are on our lab wiki
(\url{https://osf.io/cxwyz/wiki/home/}). Videos were processed using
Sony Vegas and in-house video-editing scripts. Footage was aligned in a
single, multi-camera view before manual language annotation in Datavyu.
Audio recordings were initially processed by LENA proprietary software,
which segments and diarizes each audio file; this output was then
converted to CLAN format for further processing and manual annotation
(MacWhinney \& Wagner, 2010). Through in-house scripts, long periods of
silence were demarcated in these CLAN files (e.g.~when the audio vest
was removed or during naps). The CLAN files were then used for manual
language annotation.

\subsection{Language Annotation}\label{language-annotation}

Recordings were annotated by trained researchers. The \enquote{sparse
annotation} entailed marking each concrete noun heard by the child. This
included words directed to or easily overheard by the child (e.g.~words
directed at a sibling next to the infant), but not distant or background
language (e.g.~background television). We operationalized
\enquote{object words} as concrete, imageable nouns (e.g.~shoe, arm).
For each object word, we included the word (as said by the speaker, e.g.
\enquote{teethies}) and lemma (i.e.~dictionary form, e.g.
\enquote{tooth}), along with three properties: utterance-type, object
presence, and talker. Utterance-type classified each object word
utterance as declarative, question, imperative, reading, singing,
short-phrase, or unclear. Short-phrase utterances include words in
isolation and short, simple noun phrases (e.g. \enquote{the red ball} or
\enquote{kitty's paw}). Object-presence was a binary measure of whether
the object was present and attended to. Lastly, the word's talker was
recorded, including live interlocutors and electronics: mother, brother,
toy, etc.

We assessed intercoder reliability on a random contiguous 10\% of the
annotations in each file. \textbf{add reliability analysis}

\begin{table}[tbp]
\begin{center}
\begin{threeparttable}
\caption{\label{tab:measures-tab}Derived count measures}
\small{
\begin{tabular}{ll}
\toprule
Measure & \multicolumn{1}{c}{Derived Count}\\
\midrule
Quantity & Noun tokens, Noun types\\
Speaker & Nouns from Mother, Nouns from Father, Unique Speakers\\
Utterance Type & Nouns in Declarative, Imperative, Question, Short-Phrase, Reading, \& Singing Utterances\\
Object Presence & Nouns said when the referent was present and attended to\\
\bottomrule
\end{tabular}
}
\end{threeparttable}
\end{center}
\end{table}

\section{Results}\label{results}

\subsection{Analysis Plan}\label{analysis-plan}

Based on the coding scheme above, we derived 12 count measures from each
recordings' annotations for each child (n=44), recording-type (audio,
video), and month (six, seven). See Table \ref{tab:measures-tab}. We
then averaged the data from month six and seven to increase the
precision of our input estimates, and since we have no
theoretically-motivated reason to predict input differences across this
4 week span (i.e.~there are no developmental or linguistic milestones
typically achieved at 6-7 months.) We also normalized the count measures
by recording length; further details are below. While we initial
anticipated conducting multi-level models with fixed effects of
recording-type and random subject-level effects, nearly all such models
revealed highly non-normal residuals (by visual inspection and Shapiro
Test), limiting interpretation across measures, even when
log-transformed. Thus, we instead report a simple set of nonparametric
analyses below.

For all recording-type comparisons, we look at whether our measures
\emph{differed} significantly (by two-tailed, paired Wilcoxon Test), and
\emph{correlated} significantly (by Kendall Rank Correlation) across the
given groups. This approach lets us compare, e.g., whether the
proportion of declaratives is indistinguishable in our audio and video
recordings independently of whether these values are correlated across
recording-types. We applied Holm's p-value adjustment for multiple
comparisons (\textbf{ADD REF:}Holm, 1979), for the set of 12 Wilcoxon
tests, and the set of 12 Kendall Correlations.

\subsection{Count Measures, Audio-
vs.~Video-recordings}\label{count-measures-audio--vs.video-recordings}

Before assessing how our 12 measures of noun input scaled between
hour-long video-recordings and daylong audio-recordings, we analyzed
recording lengths. Modally, videos were an hour (62 min, \emph{M}=60.79
min, SD=6.31, R=27.9-74.9 min), and audio-recordings were 16 hours (960
min, \emph{M}=858.41 min, SD=119.41, R=635-960 min), the maximum
capacity of the LENA device. While audio-recordings began when children
awoke, we further estimated the onsets and offsets of daytime naps by
removing the \enquote{silent} portions of the recordings (see Methods).
This provided an estimated upper-limit on infants' awake
(i.e.~non-silent) time (Mode = 654 min., \emph{M} = 603min, SD=106.8,
R=385.2-951 min ). Our estimates comported with established norms for
6--8-month-olds in the US (\textbf{ADD REF:} Mandel et al, 2010), which
are 180 minutes of daytime sleep, and 600 minutes of nighttime sleep.
Infants were always awake during video recordings (save one infant, who
fell asleep before the recording-hour ended; that video was stopped at
sleep onset).

\begin{figure}
\centering
\includegraphics{sixseven_simplepapaja_files/figure-latex/gr-derived-counts-67-diff-1.pdf}
\caption{\label{fig:gr-derived-counts-67-diff}Counts A vs.~V}
\end{figure}

\begin{figure}
\centering
\includegraphics{sixseven_simplepapaja_files/figure-latex/gr-derived-counts-67-corr-1.pdf}
\caption{\label{fig:gr-derived-counts-67-corr}Count Correlations, A vs.~V}
\end{figure}

To examine how the hour-long video data \enquote{scale} to day-length
data descriptively, we first divided the 12 count metrics from the
videos by those from the audio-recordings for each child, to derive
\enquote{video-fraction} scores. This showed that the video-recordings
were \textasciitilde{}0.07 of the length of audio-recordings, or 0.10 of
the length if only \enquote{non-silent} portions of the audio-recording
are included. However, rather than a concomitant 10-fold decrease in our
count metrics (as would be expected if the video captured a
\enquote{representative} hour of the day), the fractions averaged to
across measures; see Table @ref(tab:vboost\_table) \textbf{FIX INF
VALUES}. Thus, by and large, videos had a denser concentration of nouns
across our measures than did the audio recordings.

\begin{table}[tbp]
\begin{center}
\begin{threeparttable}
\caption{\label{tab:vboost-table}Video Fractions}
\begin{tabular}{ll}
\toprule
Measure & \multicolumn{1}{c}{Mean (SD)}\\
\midrule
Minutes & 0.07 (0.01)\\
Awake minutes & 0.1 (0.02)\\
Types & 0.31 (0.13)\\
Tokens & 0.25 (0.15)\\
Speakers & 0.43 (0.2)\\
Mother & 0.32 (0.22)\\
Father & 0.13 (0.26)\\
Declaratives & 0.19 (0.09)\\
Questions & 0.33 (0.16)\\
Imperatives & 0.27 (0.23)\\
Singing & Inf (1.46)\\
Reading & Inf (2.76)\\
Short phrases & 0.3 (0.25)\\
Object co-presence & 0.34 (0.28)\\
\bottomrule
\end{tabular}
\end{threeparttable}
\end{center}
\end{table}

\begin{table}[tbp]
\begin{center}
\begin{threeparttable}
\caption{\label{tab:propna-missing-tables}Proportion of zero values}
\small{
\begin{tabular}{lllllll}
\toprule
Video: Mothers & \multicolumn{1}{c}{Video: Fathers} & \multicolumn{1}{c}{Video: Imperatives} & \multicolumn{1}{c}{Audio: Singing} & \multicolumn{1}{c}{Video: Singing} & \multicolumn{1}{c}{Audio: Reading} & \multicolumn{1}{c}{Video: Reading}\\
\midrule
0.09 & 0.51 & 0.02 & 0.02 & 0.11 & 0.16 & 0.34\\
\bottomrule
\end{tabular}
}
\end{threeparttable}
\end{center}
\end{table}

We computed video-fractions (rather than the reciprocal,
i.e.~audio/video) because there were more zero values for videos than
audio-recordings (e.g.~instances when children did not hear any nouns
sung), rendering more undefined values. Indeed, \textgreater{}1/3 of
children did not hear nouns in reading or from fathers on videos in
either month. See Table \ref{tab:propna-missing-tables}.

We next normed our count values by the number of minutes in each. For
example, if an infant heard 500 noun-tokens in 800 minutes of non-silent
audio-recording, and 200 in 60 minutes of videos, this was normed to .62
and 3.3 noun-tokens per minute, respectively. Zero values are retained
within the normed count
data.\footnote{One infant's zero value was excluded from 'father' measures (as in \\ref{tab:propna-missing-tables}); this infant had no father.}

\begin{table}[tbp]
\begin{center}
\begin{threeparttable}
\caption{\label{tab:normtable}Count values, normed by number of minutes in recording}
\small{
\begin{tabular}{ll}
\toprule
Measure & \multicolumn{1}{c}{Inflation (normed)}\\
\midrule
Object presence & 2.90\\
Mother & 3.00\\
Father & NA\\
Declaratives & 1.90\\
Questions & 3.10\\
Short phrases & 2.50\\
Singing & 2.30\\
Reading & 2.90\\
Imperatives & 2.60\\
Type count & 3.00\\
Token count & 2.30\\
Speaker count & 3.90\\
\bottomrule
\end{tabular}
}
\end{threeparttable}
\end{center}
\end{table}

With the normed data, 11/12 of our metrics occured at significantly
lower rates in audio recordings than video recordings
(adjusted-p\textless{}.05). The remaining metric, number of nouns from
fathers, was statistically indistinguishable across recording types
(adjusted-p\textgreater{}.05). Thus, overall, per unit time, infants
heard less noun input across our metrics of quantity, talker,
utterance-type and object presence in audio recordings than in videos.

Looking next at correlations, we find that 10/12 metrics correlated in
audio vs.~video data; nouns per minute heard from fathers and in singing
did not. The size of the correlations was moderate (excluding the two
non-significant metrics, \emph{M} = 0.44, 0.27 - 0.57, all
adjusted-p\textless{}.05). See Table \ref{tab:normtable} and Figures
\ref{fig:gr-derived-normcounts-diff} and
\ref{fig:gr-derived-normcounts-corr}.

\begin{figure}
\centering
\includegraphics{sixseven_simplepapaja_files/figure-latex/gr-derived-normcounts-diff-1.pdf}
\caption{\label{fig:gr-derived-normcounts-diff}Normalized variable counts}
\end{figure}

\begin{figure}
\centering
\includegraphics{sixseven_simplepapaja_files/figure-latex/gr-derived-normcounts-corr-1.pdf}
\caption{\label{fig:gr-derived-normcounts-corr}Normalized count
correlations}
\end{figure}

\section{Exploratory Analyses over Utterance-Types and
Nouns}\label{exploratory-analyses-over-utterance-types-and-nouns}

We conclude the results section with two sets of highly exploratory
analyses, at the utterance level, and at the word level. The
utterance-type analysis is based on the unanticipated observation that
while declaratives and questions made up \textgreater{}2/3 of the input
for each recording-type, the videos appeared to contain relatively more
questions and fewer declaratives (See Fig
\ref{fig:gr-derived-counts-67-diff} and Fig
\ref{fig:gr-derived-normcounts-diff}). To test this statistically, we
converted the six utterance-type counts to proportions (i.e.~number of
nouns heard in declaratives over total noun tokens) for each
recording-type. Wilcoxon tests of each utterance type in audio-
vs.~video-recording (corrected for multiple comparisons as before)
revealed that indeed, declaratives and questions (but not the four other
utterance-types) occured at different rates in video and audio
recordings (adjusted-p\textless{}.05), with audio-recordings showing
relatively fewer questions (\(M_{video}\)=0.26, \(M_{audio}\)=0.19) and
more declaratives than videos (\(M_{video}\)=0.40, \(M_{audio}\)=0.50).
See figure \ref{fig:gr-ut-count-collapsed}.

\begin{figure}
\centering
\includegraphics{sixseven_simplepapaja_files/figure-latex/gr-ut-count-collapsed-1.pdf}
\caption{\label{fig:gr-ut-count-collapsed}Utterance Type Proportions}
\end{figure}

The word level analysis aims to provide a first-pass characterization of
whether audio and video recordings captured the same nouns and the same
relative frequencies across words and families. The distribution of
nouns in our recordings was zipfian: of the 5801unique object words
(3137 lemmas) heard across months and recording types, only 2482 (960
lemmas) were heard more than once (see Figures \ref{fig:zipfian} and
\ref{fig:top-100-logspace}).

\begin{figure}
\centering
\includegraphics{sixseven_simplepapaja_files/figure-latex/zipfian-1.pdf}
\caption{\label{fig:zipfian}Zipfian word frequency distributions}
\end{figure}

\begin{figure}
\centering
\includegraphics{sixseven_simplepapaja_files/figure-latex/top-100-logspace-1.pdf}
\caption{\label{fig:top-100-logspace}Top 100 words in log space}
\end{figure}

We examined the top 100 most frequent nouns from audio- and
video-recordings (n=136 due to ties, n=68 without words that occurred
zero times in one recording-type). Frequency across recording-types
correlated significantly (Kendall's tau: 0.39, p\textless{}.0001,) even
with zero-frequency words included (Kendall's tau: 0.25,
p\textless{}.0001; see Figure \ref{fig:top100-corr-rectype}).

\begin{figure}
\centering
\includegraphics{sixseven_simplepapaja_files/figure-latex/top100-corr-rectype-1.pdf}
\caption{\label{fig:top100-corr-rectype}Top 100 words correlations by
recording type}
\end{figure}

Finally, looking at just the top ten words by recording-type, we find
that the top audio words were far more common across families than the
top video words were (see Figure \ref{fig:top10noun-freq}). While four
of the top 10 words overlapped (baby, book, mouth, toes), the frequency
of the top audio words was roughly 3-fold that of the top video words,
again reflecting the higher density of video-recordings (which were 1/10
the length of audio recordings on average). Taken together, this
exploratory analysis suggests that daylong audio-recordings appear to
render more stable high-frequency words across families and than do
video-recordings.

\begin{figure}
\centering
\includegraphics{sixseven_simplepapaja_files/figure-latex/top10noun-freq-1.pdf}
\caption{\label{fig:top10noun-freq}Top 10 words by month and recording type}
\end{figure}

\section{Discussion}\label{discussion}

Our results can be distilled to three key findings. First, infants heard
relatively more nouns in the video recordings than in the audio
recordings. Per minute, infants heard \textasciitilde{}2-5x more noun
input across our quantity, speaker, utterance-type, and object-presence
metrics when they and their caretakers were video-recorded for an hour
versus audio-recorded for a day. Second, while our metrics correlated
across audio- and video-recordings, the relative rates of the most
prevalent utterance types, and the quantity of unattested data varied
across them. Finally, while the highest frequency words across recording
types largegly overlapped and correlated (and exhibited Zipfian
frequency distributions), top words from the daylong audio-recording
appear to better represent the noun input across families.

Our comparison across recording-types highlighted many differences
across our noun input measures, even with family and age held constant.
Our quantity results also conceptually replicate and extend those of
Tamis-Lemonda et al (2017). Despite numerous methodological differences
(recording lengths, experimenter presence, infant age, word class
analyzed), both studies find that parent talk per unit time is
significantly higher in shorter recordings. While the difference they
find is less extreme numerical (roughly 1.5-2x the number of types and
tokens in the longer vs.~shorter recording compared to our 2-3-fold
difference), this general pattern appears robust across our very
different sampling methods. Taken together, these results converge in
suggesting that shorter recordings elicit denser caregiver talk.

We find consistently more object co-presence in video- than in
audio-recordings. This may be because the video recordings truly had
more object presence (i.e.~infants mostly stayed in 1-2 rooms,
interacting with caregivers and objects at hand). Alternatively, or
additionally, it may be the case that there are more ambiguous cases of
\enquote{object co-presence} in audio recordings than video recordings,
which were in turn annotated as \enquote{not present} at higher rates.
Given the XX rates of agreement, we find it more likely that this
reflects a true difference between situations that arise during
daylong-audio vs.~hourlong-video recordings. Insofar as object presence
is linked with early word learning (Bergelson \& Aslin, 2017), a more
extensive understanding of what modulates it is an important issue left
to future work.

We did not anticipate that the top utterance-types would vary by
recording-type. That is, while questions and declaratives made up the
majority of the input for each recording-type at each month, videos had
relatively more questions and fewer declaratives. This is key example of
methodological choices potentially influencing language acquisition
theories: base rates of interrogatives taken from videos would inflate
estimates of auxiliary verbs in the early input. Indeed, previous work
has noted that published studies vary in whether they find links between
questions (yes/no and wh-) in the input, and children's early
productions, with developmental level of the child invoked to explain
differences across studies (Barnes et al, 1983; see discussion in
Huttenlocher et al, 2002). Here we add the possibility that
recording-type too may contribute to the base-rates of questions in the
input, even with age kept constant.

\subsection{Top Words}\label{top-words}

The pattern across recording-types suggests to us that parents behaved
naturally during recordings, but that \enquote{natural} behavior
differed by recording context. This is consistent with a point made by
Suskind et al (2013) regarding an intervention: \enquote{sustaining
increased talk for a 10-hr recording day is much less likely than being
on best behavior during {[}a{]} 1-hr videotaped session\ldots{}} While
their work aimed to encourage caretakers to talk more, the point stands
for our goals of observing infants' typical input. We add to their
suggestion that shorter video-recording itself may elicit certain kinds
of interactions, separate from deliberate intent or lack thereof on
caretakers' part.

Indeed, the kinds of everyday interactions we captured in daylong audio
recordings (family members rushing to get out the door or get meals on
the table, sibling quibbles, etc.) tended to \enquote{feel} more
natural. Families likely simply found it easier to go about their day
freely with infants in a special vest than with a camera on their head,
and a camcorder in the corner. Lending some support that equipment
prominence matters, both \enquote{hat} and \enquote{camera} are in the
top 10 words from video-recordings each month; no analogous nouns
(e.g.~vest, recorder) topped the frequency rankings in our audio
recordings (see Figure XX).

Our interpretation of the present results is that findings based on
relatively short video-recordings overestimate young infants' typical
noun input, and that extrapolation based on daylong audio recordings
likely better represents infants' quotidien experiences. This
underscores our third main result: that the conclusions one would draw
about which words are most common in young infants' language input
differed in their robustness across families by recording type. That is,
the top audio words were all heard by \(\geq 75\%\) of the families we
recorded; only one of the top 10 video words (\enquote{hat}) was this
common across families. This is true even though the video words had
greater quantities of nouns per unit time; the top audio words only
occured 2-4 times more often than the top video words (despite a 10-fold
increase in awake recording time).

\subsection{Limitations}\label{limitations}

Given the technical limitation that available infant-friendly
video-recorders have a shorter battery life than audio-recorders at
present, we cannot conclusively separate the effects of modality and
length. That is, had we only audio-recorded for an hour, or recorded
video all day, we may have obtained equivalent results across recording
modalities. Such a comparison awaits technological progress.

A further limitation is the likely influence of self-selection into the
study: many parents are unwilling to invite researchers to record their
infants' interactions. Relatedly, our convenience sample does not
reflect the broader demographics of the US (let alone other cultures or
populations), and as such this work merits extension to other
populations before conclusive generalizations about sampling methodology
can be made (cf Bergelson et al, under review).

\subsection{Conclusions}\label{conclusions}

Understanding what infants learn from is a key part in understanding
what and how they learn at all. Here we have taken first steps in
understanding how two different data collection approaches may influence
our conclusions about early linguistic input. We find that even
naturalistic observer-free video-recordings appear to inflate language
input. Without knowing how our sampling methods may be limiting us in
principle, we necessarily limit our ability to adequately model infant
learning. We\ldots{}

Notes to self: Might want to reorder points, and make point 3 a little
more general, some overarching version of the differences in conclusions
Discuss zeros/unattested datapoints

\newpage

\section{References}\label{references}

\setlength{\parindent}{-0.5in} \setlength{\leftskip}{0.5in}

\hypertarget{refs}{}
\hypertarget{ref-bergelson2017nature}{}
Bergelson, E., \& Aslin, R. N. (2017). Nature and origins of the lexicon
in 6-mo-olds. \emph{Proceedings of the National Academy of Sciences},
\emph{114}(49), 12916--12921.

\hypertarget{ref-bergelson2013acquisition}{}
Bergelson, E., \& Swingley, D. (2013). The acquisition of abstract words
by young infants. \emph{Cognition}, \emph{127}(3), 391--397.

\hypertarget{ref-bergmann2016discriminability}{}
Bergmann, C., Cristia, A., \& Dupoux, E. (2016). Discriminability of
sound contrasts in the face of speaker variation quantified. In
\emph{Proceedings of the 38th annual meeting of the cognitive science
society} (Vol. 510).

\hypertarget{ref-brent2001role}{}
Brent, M. R., \& Siskind, J. M. (2001). The role of exposure to isolated
words in early vocabulary development. \emph{Cognition}, \emph{81}(2),
B33--B44.

\hypertarget{ref-cartmill2013quality}{}
Cartmill, E. A., Armstrong, B. F., Gleitman, L. R., Goldin-Meadow, S.,
Medina, T. N., \& Trueswell, J. C. (2013). Quality of early parent input
predicts child vocabulary 3 years later. \emph{Proceedings of the
National Academy of Sciences}, \emph{110}(28), 11278--11283.

\hypertarget{ref-debaryshe1993joint}{}
DeBaryshe, B. D. (1993). Joint picture-book reading correlates of early
oral language skill. \emph{Journal of Child Language}, \emph{20}(2),
455--461.

\hypertarget{ref-dudley2009pathologizing}{}
Dudley-Marling, C., \& Lucas, K. (2009). Pathologizing the language and
culture of poor children. \emph{Language Arts}, \emph{86}(5), 362--370.

\hypertarget{ref-fernald2013ses}{}
Fernald, A., Marchman, V. A., \& Weisleder, A. (2013). SES differences
in language processing skill and vocabulary are evident at 18 months.
\emph{Developmental Science}, \emph{16}(2), 234--248.

\hypertarget{ref-hart1995meaningful}{}
Hart, B., \& Risley, T. R. (1995). \emph{Meaningful differences in the
everyday experience of young american children.} Paul H Brookes
Publishing.

\hypertarget{ref-hoff2002children}{}
Hoff, E., \& Naigles, L. (2002). How children use input to acquire a
lexicon. \emph{Child Development}, \emph{73}(2), 418--433.

\hypertarget{ref-lidz2003infants}{}
Lidz, J., Waxman, S., \& Freedman, J. (2003). What infants know about
syntax but couldn't have learned: Experimental evidence for syntactic
structure at 18 months. \emph{Cognition}, \emph{89}(3), 295--303.

\hypertarget{ref-macwhinney2001emergentist}{}
MacWhinney, B. (2001). Emergentist approaches to language.
\emph{TYPOLOGICAL STUDIES IN LANGUAGE}, \emph{45}, 449--470.

\hypertarget{ref-macwhinney2010transcribing}{}
MacWhinney, B., \& Wagner, J. (2010). Transcribing, searching and data
sharing: The clan software and the talkbank data repository.
\emph{Gesprachsforschung: Online-Zeitschrift Zur Verbalen Interaktion},
\emph{11}, 154.

\hypertarget{ref-michaels2013commentary}{}
Michaels, S. (2013). Commentary: Déjà vu all over again: What's wrong
with hart \& risley and a`` linguistic deficit'' framework in early
childhood education? \emph{Learning Landscapes}, \emph{7}(1), 23--41.

\hypertarget{ref-noble2005neurocognitive}{}
Noble, K. G., Norman, M. F., \& Farah, M. J. (2005). Neurocognitive
correlates of socioeconomic status in kindergarten children.
\emph{Developmental Science}, \emph{8}(1), 74--87.

\hypertarget{ref-oller2010automated}{}
Oller, D. K., Niyogi, P., Gray, S., Richards, J., Gilkerson, J., Xu, D.,
\ldots{} Warren, S. (2010). Automated vocal analysis of naturalistic
recordings from children with autism, language delay, and typical
development. \emph{Proceedings of the National Academy of Sciences},
\emph{107}(30), 13354--13359.

\hypertarget{ref-rost2010finding}{}
Rost, G. C., \& McMurray, B. (2010). Finding the signal by adding noise:
The role of noncontrastive phonetic variability in early word learning.
\emph{Infancy}, \emph{15}(6), 608--635.

\hypertarget{ref-roy2015predicting}{}
Roy, B. C., Frank, M. C., DeCamp, P., Miller, M., \& Roy, D. (2015).
Predicting the birth of a spoken word. \emph{Proceedings of the National
Academy of Sciences}, \emph{112}(41), 12663--12668.

\hypertarget{ref-taine1876note}{}
Taine, H. (1876). Note sur l'acquisition du langage chez les enfants et
dans l'espèce humaine. \emph{Revue Philosophique de La France et de
L'Etranger}, 5--23.

\hypertarget{ref-tamis2017power}{}
Tamis-LeMonda, C. S., Kuchirko, Y., Luo, R., Escobar, K., \& Bornstein,
M. H. (2017). Power in methods: Language to infants in structured and
naturalistic contexts. \emph{Developmental Science}.

\hypertarget{ref-tomasello2000young}{}
Tomasello, M. (2000). Do young children have adult syntactic competence?
\emph{Cognition}, \emph{74}(3), 209--253.

\hypertarget{ref-vandam2016homebank}{}
VanDam, M., Warlaumont, A. S., Bergelson, E., Cristia, A., Soderstrom,
M., De Palma, P., \& MacWhinney, B. (2016). HomeBank: An online
repository of daylong child-centered audio recordings. In \emph{Seminars
in speech and language} (Vol. 37, pp. 128--142). Thieme Medical
Publishers.

\hypertarget{ref-weisleder2013talking}{}
Weisleder, A., \& Fernald, A. (2013). Talking to children matters: Early
language experience strengthens processing and builds vocabulary.
\emph{Psychological Science}, \emph{24}(11), 2143--2152.

\hypertarget{ref-williams1937analytical}{}
Williams, H. M. (1937). An analytical study of language achievement in
preschool children. \emph{University of Iowa Studies in Child Welfare},
\emph{13}, 9--18.

\hypertarget{ref-yurovsky2013statistical}{}
Yurovsky, D., Smith, L. B., \& Yu, C. (2013). Statistical word learning
at scale: The baby's view is better. \emph{Developmental Science},
\emph{16}(6), 959--966.






\end{document}
